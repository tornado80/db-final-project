\documentclass{article}

\usepackage[margin=1in]{geometry}
\usepackage{listings}
\usepackage{xcolor}
\usepackage{qtree}
\usepackage{fancyhdr}
\usepackage{amsfonts}
\usepackage{amsmath}
\usepackage{amssymb}
\usepackage{mathtools}
\usepackage{setspace}
\usepackage{xepersian}

\pagestyle{fancy}

\settextfont{XB Niloofar}
\setlatintextfont{Code New Roman}

% Fixture for Xepersian 23 bug in setting persian math digit fonts
%\ExplSyntaxOn \cs_set_eq:NN \etex_iffontchar:D \tex_iffontchar:D \ExplSyntaxOff
%\setmathdigitfont{XB Niloofar}

\newcommand{\me}{امیرحسین رجبی}
\newcommand{\homework}{
	پروژه پایانی پایگاه داده (بخش \lr{ETL})
}
\newcommand{\studentNumber}{9813013}
\newcommand{\meAndStudentNumber}{\me \vspace{1pt} (\studentNumber)}
\newcommand{\instructor}{دکتر غیبی}

\fancyhead[C]{\homework}

\title{\textbf{\homework}}
\author{\meAndStudentNumber}
\date{}

\definecolor{codegreen}{rgb}{0,0.6,0}
\definecolor{codegray}{rgb}{0.5,0.5,0.5}
\definecolor{codepurple}{rgb}{0.58,0,0.82}
\definecolor{backcolour}{rgb}{0.95,0.95,0.92}

\lstdefinestyle{mystyle}{
	backgroundcolor=\color{backcolour},   
	commentstyle=\color{codegreen},
	keywordstyle=\color{magenta},
	numberstyle=\tiny\color{codegray},
	stringstyle=\color{codepurple},
	basicstyle=\ttfamily\footnotesize,
	breakatwhitespace=false,         
	breaklines=true,                 
	captionpos=b,                    
	keepspaces=true,                 
	numbers=left,                    
	numbersep=5pt,                  
	showspaces=false,                
	showstringspaces=false,
	showtabs=false,                  
	tabsize=2
}

\lstset{style=mystyle, language=bash}

\begin{document}

\maketitle

\doublespacing

\section*{نصب}

\begin{latin}
\begin{lstlisting}
	cd etl/src
	python3 -m venv env
	source env/bin/activate
	pip install -r requirements.txt
	python main.py -h
\end{lstlisting}
\end{latin}

\section*{نحوه استفاده}

به کمک کتابخانه 
\lstinline|argparse|
یک رابط 
\lr{CLR}
طراحی شده است که میتوان صفحه 
\lr{help}
را به کمک دستور
\lr{\lstinline|python main.py -h|}
آن را مشاهده کرد. به طوری کلی ورودی های ضروری حداقل مشخصات لازم برای اتصال به پایگاه داده مبدا و مقصد است. (یعنی نام، نام کاربر پایگاه داده و رمز آن) اما در صورت محلی
\footnote{\lr{non local or remote}}
نبودن پایگاه داده، آدرس سرور میزبان
\footnote{\lr{host}}
و درگاه 
\footnote{\lr{port}}
می توانند مشخص شوند. همچنین میتوان اسکیما 
\footnote{\lr{schema}}
که جداول در آن قرار دارند را مشخص کرد.  مثلا دستور زیر یک دستور معتبر است:
\begin{latin}
\begin{lstlisting}
	python main.py -s db1 user1 pass1 -d db2 user 2 pass2 -ss schema1 -ds schema2 -sh host1 -sp port1 -dh host2 -dp port2
\end{lstlisting}
\end{latin}

\section*{ساختار کد}

\end{document}